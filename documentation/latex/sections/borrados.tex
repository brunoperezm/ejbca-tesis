\section{Modelo C4 y herramientas de diagramación}
\label{sec:diag_c4}

En este trabajo la arquitectura del sistema se documenta mediante el \emph{modelo C4}, una propuesta de Simon Brown que introduce claridad al describir software en distintos niveles de abstracción: \emph{Context}, \emph{Container}, \emph{Component} y \emph{Code}\cite{brown_c4_model}.  

\subsection{Ventajas del modelo C4}

En contraste con UML, C4 utiliza un conjunto limitado de conceptos (persona, sistema, contenedor y componente) que simplifica la comunicación con públicos técnicos y no técnicos, reduce la curva de aprendizaje y evita la proliferación de diagramas poco mantenibles. Estas características lo convierten en una herramienta idónea para sesiones de diseño colaborativo, incorporación de nuevos integrantes y actividades de \emph{threat modelling}\cite{brown_c4_model}.

\subsection{Diagrama como código: Structurizr DSL}

Para materializar las vistas C4 se empleó \textbf{Structurizr DSL}, un lenguaje declarativo que describe modelo, vistas y estilos en texto plano\cite{structurizr_dsl}. Esta aproximación \emph{diagram-as-code} facilita:

\begin{itemize}
    \item control de versiones y revisión de cambios en el mismo flujo que el código fuente,
    \item generación reproducible de diagramas en \emph{pipelines} CI/CD,
    \item automatización de revisiones de arquitectura.
\end{itemize}

Los archivos \texttt{workspace.dsl} incluidos en el repositorio pueden renderizarse localmente con \textit{Structurizr CLI} o mediante \textit{Structurizr Lite}\cite{structurizr_lite}, garantizando consistencia entre documentación y estado real del sistema.

\subsection{Emisión de Certificados}

Una vez todo el sistema estuviera implementado, en el día a día, para que un trabajador nuevo, o uno que se le hubiera revocado su certificado, cualquiera sea la razón, pueda ingresar a un servidor, el primer paso será el de emitirle un certificado nuevo.

Este paso es probablemente el que más se puede ver influenciado por la política interna que se decida implementar, la propuesta nuestra se puede observar en la figura \ref{fig:emision_certificados} en el caso de un usuario nuevo.

\begin{figure}[H]
    \centering
    \includegraphics[width=0.7\textwidth]{fig/emision_certificados.png}
    \caption{Emisión certificados para usuario nuevo}
    \label{fig:emision_certificados}
\end{figure}

En el diagrama \ref{fig:emision_certificados} aparece un miembro nuevo, no mencionado anteriormente, identificado como \textit{Coordinador}, rol que tendrá un miembro con la posición para supervisar y organizar el trabajo, por lo que tendrá conocimiento de los trabajadores, y si deben tener acceso o no a un servidor para realizar que tarea. 

El Coordinador le solicitara a la persona encargada de la tarea de Oficial de Registro, encargada de la AR (incluso pueden ser la misma persona), que le de un certificado al usuario nuevo, y en este caso, el Oficial de Registro le dará al usuario un instructivo con los pasos a realizar en su equipo.

Dichos pasos consisten en generar un cert.csr, habiendo completado un csr.conf con su información personal, y firmado el mismo con un par de claves generado dentro de un SoftHSM, indicado en el instructivo todos los pasos necesarios para llegar a dicho fin.

Una vez generado el csr.conf, se lo entrega al Oficial de Registro, quien tras examinar la validez del mismo y de su información, lo da de alta en el EJBCA, y genera un certificado, que le es entregado al usuario, e informado al Coordinador de que la tarea fue completado.

En este esquema, se asume que el usuario nuevo sera capaz de seguir los pasos indicados en el instructivo, y no se toma en cuenta los pasos siguientes que deberá tomar para poder usar el certificado para autenticarse contra un servidor. 

Aquí es donde surgen muchas variantes para la propuesta, tales como entregarle al usuario programas que realicen las tareas necesarias en su equipo para las diferentes etapas, o que deba llevar su equipo con el Oficial de Registro o el Coordinador, para que sean ellos quienes operen sobre su equipo para dejarlo en condiciones.


