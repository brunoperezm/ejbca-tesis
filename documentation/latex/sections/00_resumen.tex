\chapter*{Resumen del trabajo efectuado}
En el Laboratorio de Redes y Ciberseguridad  se configuró un servidor con VMware ESXi, para hacer uso de maquinas virtuales, con el que se desarrolló un entorno inspirado en el de Prosecretaría de Informática.

Esto se hizo con el objetivo de implementar y comprobar el funcionamiento de un sistema de autenticación a los servidores vía SSH, haciendo uso de un Laboratorio de Firma Electrónica, para que centralice la emisión y comprobación de validez de certificados digitales basados en el estándar X.509, con el cual hacer mas robusta la seguridad y escalabilidad de los procedimientos hoy utilizados.

Haciendo uso de herramientas Open Source, se consiguió implementar un sistema en el cual la autenticación de cada usuario, en cada servidor, esta atado a un sistema de clave pública y privada, en el cual se debe tener emitido un certificado digital valido, que puede ser revocado en cualquier momento por una autoridad competente, sin que sea necesario intervenir en ningún servidor manualmente para dar acceso o negarlo, consiguiendo una gran escalabilidad y control de acceso.

\vspace{.5cm}

\textbf{Palabras clave:} Computer Security, Access Control, Cryptography, Public Key.

