\chapter{Abreviaturas}\label{cap:glosario}
% Ver si se cambia a tabla de abreviaturas

Debido al que público objetivo de este proyecto es uno que se considera conocedor del tema, no es necesario hacer uso de un glosario para la explicación de los conceptos mencionados, sin embargo, para evitar posibles errores de interpretación, se lo desarrolla con el objetivo de tener una referencia a los acrónimos utilizados.



\begin{longtable}{|c|l|}
\hline
\textbf{Sigla} & \textbf{Significado} \\
\hline
\endfirsthead

\hline
\endfoot

\hline
Infraestuctura TI & Infraestructura de Tecnología de Información \\
SSH & Shell Seguro \\
PSI & Pro Secretaria Informática \\
UNC & Universidad Nacional de Córdoba  \\
MitM & Ataque de intermediario \\
MFA & Autenticación Multifactor \\
VM & Máquina Virtual \\
PKI & Infraestructura de Clave Pública \\
PKIX & Infraestructura de Clave Pública basada en certificados X.509 \\
IETF & Internet Engineering Task Force \\
RFC & Request For Comments \\
CRL & Lista de Revocación de Certificados \\
AC & Autoridad de Certificación \\
AR & Autoridad de Registro \\
AV & Autoridad de Validación \\
OCSP & Protocolo de Verificación de Certificados en Línea \\
DN & Nombres Distinguidos \\
SCEP & Protocolo Simple de Inscripción de Certificados \\
EST & Registro Sobre Transporte Seguro \\
ACME & Entorno de Gestión de Certificados Automatizado \\
CMP & Protocolo de Gestión de Certificados \\
HSM & Módulo de Seguridad de Hardware \\
SCSI/IP & Interfaz de Sistemas Informáticos Pequeños a través de protocolo IP \\
PKCS & Normas de Criptografía de Clave Pública \\
PCIe & Interconexión de Componentes Periféricos Exprés \\
API & Interfaz de Programación de Aplicaciones \\
REST & Transferencia de Estado Representacional \\
HTTP & Protocolo de Transferencia de Hipertexto \\
SOAP & Protocolo Simple de Acceso a Objetos \\
XML & Lenguaje de Marcado Extensible \\
IoT & Internet de las Cosas \\
PEM & Correo de Seguridad Mejorado \\
CRUD & Crear, Leer, Actualizar, Eliminar \\
FTP & Protocolo de Transferencia de Archivos \\
PAM & Módulos de Autenticación Conectables \\
2FA & Autenticación de Dos Factores \\
VPN & Red Privada Virtual \\
OTP & Contraseña de Un Uso \\
IP & Protocolo de Internet \\
CNFC & Fundación para la Computación Nativa en la Nube \\
RBAC & Control de Acceso Basado en Roles \\
ABAC & Control de Acceso Basado en Atributos \\
UI & Interfaz de Usuario \\
GUI & Interfaz Gráfica de Usuario \\
RRHH & Recursos Humanos \\
SO & Sistema Operativo \\
SLA & Acuerdo de Nivel de Servicio \\
MB & MegaByte \\
RAM & Memoria de Acceso Aleatorio \\
LARyC & Laboratorio de Redes y Ciberseguridad  \\
FCEFyN & Facultad de Ciencias Exactas Físicas y Naturales  \\
UUID & Identificador Único Universal \\
CVE & Vulnerabilidades y Exposiciones Comunes \\
FIPS & Estándares Federales de Procesamiento de la Información \\
VLAN & Red de Área Local Virtual \\
NIC & Tarjeta de Interfaz de Red \\
URL & Localizador Uniforme de Recursos \\
CLI & Interfaz de Línea de Comandos \\
CP & Perfiles de Certificados \\
EEP & Perfiles de Entidades Finales \\
VO & Value Object \\
SRP & Single Responsibility Principle \\
DIP & Dependency Inversion Principle \\
CN & Nombre Común \\
\hline
\end{longtable}